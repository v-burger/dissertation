\chapter{Conclusion}\label{chap:conclusion}

%Motivation
The vast majority of Internet traffic is carried by content delivery networks.
CDNs distribute popular content to caches in many geographical areas to save bandwidth by avoiding unnecessary multihop retransmission.
A high potential to bring content even closer to consumers and to reduce energy cost is achieved by caching at basestation, the user premises or at end devices.
Content Delivery Networks


%Problem Statement
However
consumption and operation cost of the system
%Consequence

%Contribution
In this monograph
chapter 2
The caches are organized in an hierarchical network, which may include an ISP cache. To assess the Internet wide potential of such approaches, we evaluate the Internet Census Dataset and provide a distribution of active IP addresses on autonomous systems.
The evaluation shows that autonomous system size in terms of active IP addresses is highly heterogeneous.
The 10 largest autonomous systems already contain 30\% of the active IP addresses.

Chapter 3
In order to evaluate the performance of cache systems with small capacities and limited upload bandwidth we develop
an analytical model based on the Erlang formula for loss networks.
Our results show that there is a high potential to increase the efficiency of the content delivery network if only a small or no ISP cache is available.
If a larger ISP cache is available the benefit of the approach highly depends on the number of caches available and their upload bandwidth.

Chapter 4

Major insights

Impact on future work
