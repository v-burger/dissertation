\section{Lessons Learned}\label{sec:aggregation:lessonslearned}
In this chapter we studied

This chapter provides a two-pronged approach to analysing the impact of changes by individual stakeholders on the overall network.

First, we provided
Based on this algorithm we analyse

Second, we propose
Our results show

Concluding from this chapter, to reduce the load on cellular networks and to cope with the increasing demand of traffic carried by mobile networks, traffic is offloaded to WiFi networks.
To even increase the available bandwidth, recent concepts consider aggregating backhaul access link capacities.
In this work an approximation of a partial sharing scheme is presented, which is used to analyze the performance of a system with multiple access links that share their bandwidth. A joint fixed point iteration of an outer and an inner composite system is used to derive the state probabilities in heterogeneous load conditions.
In parameter studies we investigate the potential of the mechanism depending on the number of cooperating systems.
Our results show that the bandwidth of an overloaded system can exceed its capacity multiple times if the cooperating systems are underutilized, especially if the number of cooperating systems is high.
By prioritizing systems, we can show that the mechanism is robust against free riders and thus provides incentives to contribute to increase the overall system capacity.

Approaches like \emph{Economic Traffic Management}~\cite{spirou2009} or \emph{Design for Tussle}~\cite{trilogy2008} could be applied to find
an acceptable tradeoff for all parties.
In Economic Traffic Management all participating entities share information in order to enable collaboration.
This collaboration allows for a joint optimisation of the tradeoff.
Design for Tussle aims to resolve tussles at run time, instead instead of design time.
This prevents the case that one actor has full control over the optimisation problem, which would likely result in the actor choosing a tradeoff only in its favour, ignoring all other participants.
