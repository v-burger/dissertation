\section{Lessons Learned}\label{sec:network:lessons_learned}
In this chapter we studied

This chapter provides a two-pronged approach to analysing the impact of changes by individual stakeholders on the overall network.

First, we provided
Based on this algorithm we analyse

Second, we propose
Our results show

Concluding from this chapter, we see that in mobile networks many different players, metrics, and tradeoffs exist.
We highlighted one example of such a tradeoff, i.e. signalling load vs. power drain and discussed the influence of the current optimisation parameters, the network timers, on another.
However, many additional tradeoffs exist.
For example, the mobile operator has to balance the use of radio resources with the number of generated signalling frequencies.
Furthermore, application providers seek to improve the user experience which usually result in a higher frequency of network polls, creating additional signalling traffic.
The high number of tradeoffs and involved actors in this optimisation problem indicate that the current optimisation technique used by operators is no longer sufficient.

Approaches like \emph{Economic Traffic Management}~\cite{spirou2009} or \emph{Design for Tussle}~\cite{trilogy2008} could be applied to find
an acceptable tradeoff for all parties.
In Economic Traffic Management all participating entities share information in order to enable collaboration.
This collaboration allows for a joint optimisation of the tradeoff.
Design for Tussle aims to resolve tussles at run time, instead instead of design time.
This prevents the case that one actor has full control over the optimisation problem, which would likely result in the actor choosing a tradeoff only in its favour, ignoring all other participants.
